\documentclass[11pt]{article}
\usepackage[utf8]{inputenc}
\usepackage[russian]{babel}
\usepackage{pgf}
\usepackage{tikz}
\usepackage{pgf-umlcd}
\title{\textbf{Генетические алгоритмы}}
\author{Ильин Андрей\\Павлович Владислав}
\date{}
\begin{document}

\maketitle

\section{Введение}

\section{Описание библиотеки}
\subsection{Общая структура}

Библиотека состоит из четырёх основных компонентов (каждый из них - шаблонный класс), каждй из которых может быть заменён по выбору пользователя. Таким образом разработанная библиотека предоставляет набор модулей, из затем составляется необходимая пользователю утилита.

\begin{tikzpicture}[show background grid]
  \begin{package}{Genetic}
    \begin{class}{Population}{3,0}
      \attribute{\# dna : BaseLinearDna}
      \operation{+ test ()}
    \end{class}

    \begin{class}{Generation}{3,-3}
    \end{class}
    
    \begin{class}{Individual}{3,-6}
    \end{class}
    
    \begin{class}{Dna}{3,-9}
    \end{class}
    
    \aggregation{Population}{generation}{1}{Generation}
    \aggregation{Generation}{individuals}{2..*}{Individual}
    \composition{Individual}{dna}{1}{Dna}
  \end{package}
\end{tikzpicture}

\subsection{Вспомогательные модули}
\begin{tikzpicture}[show background grid]
  \begin{package}{Genetic}
    \begin{abstractclass}[text width=4cm]{Individual}{0,-6}
      \attribute{\# dna : BaseLinearDna}
      \operation{+ test ()}
    \end{abstractclass}

    \begin{class}[text width=6cm]{TestIndividual}{6,-3}
      \inherit{Individual}
    \end{class}

    \begin{class}[text width=6cm]{MinSearchIndividual}{6,-6}
      \inherit{Individual}
    \end{class}
  \end{package}

\end{tikzpicture}

\end{document}
