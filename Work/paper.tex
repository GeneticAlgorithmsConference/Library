\documentclass[a4paper, oneside, 11pt]{article}
\usepackage[utf8]{inputenc}
\usepackage[russian]{babel}
\usepackage{pgf}
\usepackage{tikz}
\usepackage{pgf-umlcd}
\usepackage{amsmath}
\usepackage{mathtools}
\usepackage{tabu}
\usepackage{array}
\usepackage{multirow}
\usepackage{graphicx}
\usepackage{layout}
\textwidth = 500pt
\textheight = 750pt
\setlength{\parindent}{10cm}
\usetikzlibrary{shapes, arrows, svg.path, calc}
\newcommand\abs[1]{\left|#1\right|}
\title{\textbf{Практическое применение генетических алгоритмов}}
\author{Ильин Андрей\\Павлович Владислав\\\\Научный руководитель:\\ Лапо Анжелика Ивановна\\}
\date{}
\begin{document}

\newpage
\hoffset = -50pt
\voffset = -75pt
\maketitle
\newpage

\tableofcontents
\newpage
\parindent=1cm

\section{Введение}
\indent\parВ последнее время возникает всё больше различных оптимизационных задач, и генетические алгоритмы - идин из методов их решения. Генетический алгоритм (ГА) – это эвристический алгоритм поиска, используемый для решения задач оптимизации и моделирования путём случайного подбора и комбинирования искомых параметров, основанный на принципах естественного отбора Ч. Дарвина. ГА полезны при решении таких задач, как:  оптимизация функций и запросов в базах данных, разнообразные задачи на графах (задача коммивояжера, раскраска, нахождение паросочетаний и т. д.),  составление расписаний, игровые стратегии, а также в искусственной нейронных сетях. Теорема схем, которая обосновывает эффективность генетических алгоритмов была впервые сформулирована и доказана Джоном Холландом в 1975 году. В 1980-х компания 
General Electric начала продажу первого в мире продукта, работающего с использованием генетических алгоритмов.

\section{Цель работы}
\indent\parОсновной целью работы было создание простой во внедрении библиотеки, реализующей генетические алгоритмы. Для её достижения нами были поставлены следующие задачи:
\begin{enumerate}
\item Изучить основные принципы и области применения генетических алгоритмов.
\item Разработать  библиотеку, позволяющую легко интегрировать функции генетических алгоритмов в любое приложение. При этом сделать структуру библиотеки максимально универсальной и простой в использовании.
\item Протестировать работу библиотеки на примере нескольких приложений и оценить эффективность использования генетических алгоритмов.
\end{enumerate}

\section{Основные понятия и принципы работы генетических алгоритмов}
\subsection{Основные понятия}
Приведём понятия, используемые для описания генетических алгоритмов в [1]:
\begin{itemize}
\item Вектор — упорядоченный набор чисел, называемых компонентами вектора. Так как вектор можно представить в виде строки его координат, то в дальнейшем понятия вектора и строки считаются идентичными. Заметим, что данное определение вектора отличается от определения, принятого в математике.

\item Булев вектор — вектор, компоненты которого принимают значения из двух элементного (булева) множества, например, \{0, 1\} или \{-1, 1\}.

\item Расстояние Хемминга — число позиций, в которых соответствующие символы двух строк одинаковой длины различны.

\item Хромосома — вектор (или строка) из каких-либо чисел. Если этот вектор представлен бинарной строкой из нулей и единиц, например, 1010011, то он получен либо с использованием двоичного кодирования, либо кода Грея. Каждая позиция (бит) хромосомы называется геном.

\item Индивидуум (генетический код, особь) — набор хромосом (вариант решения задачи). Обычно особь состоит из одной хромосомы, поэтому в даль-нейшем особь и хромосома идентичные понятия.

\item Расстояние — хеммингово расстояние между бинарными хромосомами.

\item Мутация — случайное изменение одной или нескольких позиций в хромосоме. Например, 1010011 $\to$ 1010001.

\item Популяция — совокупность индивидуумов.

\item Пригодность (приспособленность) — критерий или функция, экстремум которой следует найти.
\end{itemize}
\parИз приведенных выше определений следует, что терминология ГА представляет собой синтез собственно генетических и искусственных понятий.
Так, для понятия, заимствованного из генетики, можно предъявить его искусственный (символический) аналог. Например, хромосома и строка. В
биологических системах полный генетический пакет
называется генотипом. В искусственных системах полный генетический пакет строк называется структурой. В биологических системах в процессе индивидуального
развития организма взаимодействие генотипа с окружающей средой формирует совокупность внешних признаков и свойств, называемую фенотипом.
В математическом моделировании рассматриваемая структура декодируется
с помощью множества параметров, которое в литературе иногда называ-ют альтернативным решением или точкой. Всевозможные значения пара-метров образуют пространство решений.
\subsection{Принцип работы}
\tikzstyle{decision} = 
[
	diamond,
	draw,
	fill = green!20,
	text width = 6em,
	text badly centered,
	node distance = 4cm,
	inner sep = 0pt
]
\tikzstyle{block} = 
[
	rectangle,
	draw,
	fill = blue!20,
	text centered,
	node distance = 2cm,
	rounded corners,
	minimum height = 2em
]
\tikzstyle{line} =
[
	draw,
	-latex'
]
\tikzstyle{cloud} = 
[
	rectangle,
	draw,
	fill = red!20,
	rounded corners,
	node distance = 3cm,
	minimum height = 2em
]
\tikzstyle{point} = 
[
	circle
]
\begin{tikzpicture}
\node [cloud] (start) {Генерация начальной популяции};
\node [block, below of = start] (phase1) {Вычисление приспособленности};
\node [decision, below of = phase1] (phase2) {Достигнут критерий окончания процесса?};
\node [cloud, below of = phase2] (end) {Выход из цикла};
\node at ($ (phase2.east) + (0.5,0.3) $) (phase2no) {Нет};
\node at ($ (phase2.south) - (0.5,0.2) $) (phase2yes) {Да};
\node [point] at ($ (phase2.east) + (1,0) $) (phase2nopoint) {};
\node [block] at ($ (phase2nopoint.east) + (3,-2) $) (phase3) {Выбор родителей};
\node [point] at ($ (phase2nopoint.center) - (0,2) $) (phase3point) {};
\node [block, above of = phase3] (phase4) {Кроссинговер};
\node [block, above of = phase4] (phase5) {Мутация};
\node [block, above of = phase5] (phase6) {Новая популяция};
\path [line] (start) -- (phase1);
\path [line] (phase1) -- (phase2);
\path [line] (phase2) -- (end);
\path [line] (phase3) -- (phase4);
\path [line] (phase4) -- (phase5);
\path [line] (phase5) -- (phase6);
\path [line] (phase6) -- (phase1);
\path [line] (phase2) -- (phase2nopoint.center) -- (phase3point.center)
-- (phase3);
\end{tikzpicture}
\section{Описание библиотеки}
\subsection{Общая структура}

\indent\parТри основных класса -  Generation (Поколение), Individual (Индивид) и Dna (ДНК). В них реализованы основные операции генетических алгоритмов (селекция, отбор, мутация), а также хранятся параметры для алгоритма (тип селекции, вероятность мутации и т.д.). Для внедрения генетических алгоритмов разработчику необходимо лишь создать переменную типа Generation, указав в качестве типа Individual. При этом тип индивида необходимо выбрать в зависимости от конкретной задачи из числа уже реализованных в библиотеке. Если же ни один из реализованных типов не походит к поставленной задаче, разработчик может создать свой класс, базирующийся на Individual из нашей библиотеки. Для этого класса ему необходимо определить только функцию подсчёта приспособленности и тип используемой ДНК, остальное наследуется от базового класса и не требует переопределения. Разуметься, если ДНК также необходимо специфическое, его можно реализовать самому, определив для него методы мутации и генерации. Перед использованием необходимо также задать настройки генетического алгоритма. Стоит отметить, что в ходе работы программы их можно изменять (например, увеличивать с каждым новым поколением вероятность мутации). Такой подход к работе с библиотекой даёт разработчику простор для экспериментов и позволяет найти наиболее оптимальные настройки для алгоритма в поставленной задаче. Как и положено, к библиотеке прилагается документация в форматах LaTeX, HTML и PDF. Это позволит разработчику быстро понять её структуру и внедрить её в свой проект.
\parНа данный момент в библиотеке кроме базовых классов также реализованы следующие типы ДНК:
\begin{enumerate}
\item Линейная действительная;
\item Линейная действительная с ограниченным диапазоном;
\item Линейная бинарная;
\item Древовидная ДНК выражения;
\end{enumerate}
\parВ ней также можно найти индивидов для решения таких задач, как:
\begin{enumerate}
\item Нахождение минимума / максимума функций.
\item Подбор функций под заданные значения переменных и заданный результат.
\item Подбор параметров машины для прохождения заданной трассы.
\end{enumerate}
\parБиблиотека поддерживает следующие типы скрещивания:
\begin{enumerate}
\item Дискретная рекомбинация - для каждого гена потомка случайно выбирается номер особи, ген которой он унаследует.
\item Промежуточная рекомбинация - применяется только к вещественным типам ДНК. Гены потомков вычисляются по следующей формуле: $c = a + \alpha \times b$. c - значение гена потомка, a и b соответственно значения генов первого и второго родителей.
\item Линейная рекомбинация - отличается от промежуточной тем, что множитель $\alpha$ выбирается для потомка один раз.
\item Многоточечный кроссинговер - выбираются $n$ точек разреза и происходит обмен участками хромосом, ограниченными этими точками.
\item Перетасовочный кроссинговер. В данном алгоритме особи, отобранные для кроссинговера, случайным образом обмениваются генами. Затем выбирают точку для одноточечного кроссинговера и проводят обмен частями хромосом. После скрещивания созданные потомки вновь тасуются.
\end{enumerate}
\parПоддерживаемые способы выбора родителей:
\begin{enumerate}
\item Панмиксия - выбираются дву случайные особи и скрещиваются.
\item Генотипный инбридинг - первый родитель выбирается случайным образом, а второй таким образом, что разность значений целевых функций выбранных особей минимальна.
\item Фенотипный инбридинг - первый родитель выбирается случайным образом, а второй с минимальным Хемминговым расстоянием (для строк) или Евклидовым (для векторов).
\item Генотипный аутбридинг - тоже самое, что и генотипный инбридинг, только берётся максимальное расстояние.
\item Фенотипный аутбридинг - тоже самое, что и фенотипный инбридинг, только берётся максимальная разность значений целевой функции.
\end{enumerate}
\parПоддерживается элитарный способ отбора особей в новую популяцию - в следующее поколение переходят $n$ самых лучших особей.

\parСхема библиотеки:\\


\begin{tikzpicture}[]
  \begin{package}{Genetic}
    \begin{class}{Population}{3,0}
      \attribute{\# dna : BaseLinearDna}
      \operation{+ test ()}
    \end{class}

    \begin{class}{Generation}{3,-3}
    \end{class}
    
    \begin{class}{Individual}{3,-6}
    \end{class}
    
    \begin{class}{Dna}{3,-9}
    \end{class}
    
    \aggregation{Population}{generation}{1}{Generation}
    \aggregation{Generation}{individuals}{2..*}{Individual}
    \composition{Individual}{dna}{1}{Dna}
  \end{package}
\end{tikzpicture}

\subsection{Вспомогательные модули}
\begin{tikzpicture}[]
  \begin{package}{Genetic}
    \begin{abstractclass}[text width=4cm]{Individual}{0,-6}
      \attribute{\# dna : BaseLinearDna}
      \operation{+ test ()}
    \end{abstractclass}

    \begin{class}[text width=6cm]{TestIndividual}{6,-3}
      \inherit{Individual}
    \end{class}

    \begin{class}[text width=6cm]{MinSearchIndividual}{6,-6}
      \inherit{Individual}
    \end{class}
  \end{package}
\end{tikzpicture}
\\\\
\begin{tikzpicture}[]
  \begin{package}{Genetic}
    \begin{abstractclass}[text width=4cm]{TreeIndividual}{0,0}
      \attribute{\# dna : BaseTreeDna}
      \operation{+ test ()}
    \end{abstractclass}
    \begin{class}[text width=6cm]{TreeExpressionIndividual}{6,0}
      \inherit{TreeIndividual}
    \end{class}
   \end{package}
\end{tikzpicture}
\\\\
\begin{tikzpicture}[]
  \begin{package}{Genetic}
    \begin{abstractclass}[text width=4cm]{BaseLinearDna}{0,3}
      \operation{+ mutate ()}
      \operation{+ generate ()}
      \operation{+ getDistance ()}
    \end{abstractclass}
    \begin{class}[text width=6cm]{LinearBinaryDna}{6,0}
      \inherit{BaseLinearDna}
    \end{class}
    \begin{class}[text width=6cm]{LinearRealDna}{6,3}
      \inherit{BaseLinearDna}
    \end{class}
   \end{package}
\end{tikzpicture}
\\\\
\begin{tikzpicture}[]
  \begin{package}{Genetic}
    \begin{abstractclass}[text width=4cm]{BaseTreeDna}{0,0}
      \operation{+ mutate ()}
      \operation{+ generate ()}
      \operation{+ getDistance ()}
    \end{abstractclass}
    \begin{class}[text width=6cm]{TreeExpressionDna}{6,0}
      \inherit{BaseTreeDna}
    \end{class}
   \end{package}
\end{tikzpicture}

\section{Примеры использования}
\subsection{Максимум/Минимум функции}
\indent\parДанный пример является простейшей иллюстрацией работы генетического алгоритма.  Пользователь задаёт какую-либо функцию, а задачей компьютера является нахождение её минимум/максимума. Стоит отметить, что для сложных функций результат в данном примере в большой степени зависит от мутаций, так как они помогают алгоритму находить минимум/максимум на всей области определения, а не останавливаться на локальных экстремумах. Результаты работы на различных функциях приведены в таблице ниже:

\begin{center}
\begin{tabular}{*{5}{| >{\centering\arraybackslash}p{2cm}} |}
\hline
Функция & Количество особей в популяции & Шаг алгоритма & Лучшее значение аргументов & Значение функции \\ \hline
\multirow{5}{*}{$(x + 5)^2$} & \multirow{5}{*}{100} & 1 & 1.17 &  38.0812 \\ \cline{3-5}
& & 11 & -0.454 & 20.661 \\ \cline{3-5}
& & 21 & -2.36025 & 6.96828 \\ \cline{3-5}
& & 31 & -4.704 & 0.087616 \\ \cline{3-5}
& & 41 & -4.98525 & 0.00022 \\ \hline

\multirow{5}{2cm}{$\sin{x}\cos{y} + \sin{y}\cos{x}$} & \multirow{2}{*}{100} & 1 & $x = 337.821$ $y = 251.221$ &  -0.999978 \\ \cline{3-5}
& & 11 & $x = 988.057$ $y = 380.106$ & -1 \\ \cline{2-5}
& \multirow{2}{*}{10} & 1 & $x = 835.868$ $y = 412.822$ & -0.995671 \\ \cline{3-5}
& & 11 & $x = 835.899$ $y = 412.885$ & -1 \\ \hline
\end{tabular}
\end{center}

\parТаким образом генетические алгоритмы весьма эффективны в подобных задачах. Очевидно, что линейный перебор значений с подобной точностью будет работать в разы медленнее. Дихотомия предоставляет хорошую скорость работы, но подходит только для монотонных функций.

\subsection{Подбор функции}
\indent\parВ данной задаче в качестве известны значения функции в различных точках. Программе необходимо подобрать функцию, которая в данных точках принимает значения как можно ближе к заданным. В данном случае используется древовидное ДНК. Каждый элемент - какая-либо функция ($x, \sin x$ и т. д.), поэтому данная программа хорошо подбирает только те функции, где отсутствуют различные константы. В таблице $Q = \sum_{i = 1}^{10}{\abs{f(x_i) - g(x_i)}}$, где $f(x)$ - целевая функция, а $g(x)$ - лучшая функция на текущем шаге, $x_i$ - $i$-ая точка, в которой нам известно значение функции. Анализируя точность работы программы, мы можем составить следующую таблицу (Функции представлены точно так же, как и в программе - без сокращений, для каждой известно значение ровно в десяти точках):

\begin{center}
\begin{tabular}{| >{\centering}p{2cm} | >{\centering}p{1cm} | >{\centering}p{1cm} | >{\centering}p{5cm} | c |}
\hline
Функция & Номер теста & Шаг алгоритма & Лучшая функция & Q \\ \hline
\multirow{2}{*}{$x^2$} & \multirow{2}{*}{1} & 1 & $\abs{x - x \frac{\tan{x}}{\ctg{x}}}$ & 127.56 \\ \cline{3-5}
& & 21 & $\abs{x \times \abs{\abs{\abs{x}}}}$ & 0 \\ \hline
\multirow{5}{2cm}{$x \times \tan{\cos{x}}$} & \multirow{3}{*}{1} & 1 & $\cos { x }$ &  285.325 \\ \cline{3-5}
& & 125 & $(\abs { \frac{x}{\frac{x}{\ctg { x }}} } + (x \times \cos { \abs { x } }))$ & 68.0116 \\ \cline{3-5}
& & 250 & $(\sin { \cos{ x }^{\abs{ \abs { \ctg { (x - x) } } }}} + \frac{x}{\ctg { \cos { x } }})$ & 0 \\ \cline{2-5}
& \multirow{2}{*}{2} & 1 & $\sin { \sin { x } }$ & 284.937 \\ \cline{3-5}
& & 250 & Функция указана под таблицей & 52.6177 \\ \cline{3-5}
& & 500 & $({({\frac{x}{x}} \times {x})} \times {\tan { \cos { x } }})$ & 0 \\ \hline
\end{tabular}
\end{center}

\begin{center}
$({({({({({x} + {\tan { ({({({\frac{x}{x}} + {({x} \times {x})})} + {x})} \times {\sin { x }}) }})} + {\tan { ({\tan { \tan { \frac{x}{x} } }} \times {\sin { x }}) }})} + {\tan { ({({({\frac{x}{x}} + {({x} \times {x})})} + {x})} \times {x}) }})}}$ ${+ {\tan { ({({({\frac{x}{x}} + {({({x} + {\abs { \frac{x}{x} }})} \times {x})})} + {x})} \times {({\cos { x }} - {({x} + {\tan { ({({({\frac{x}{x}} + {({x} \times {x})})} + {x})} \times {\sin { x }}) }})})}) }})} \times {\cos { x }})$
\end{center}

\subsection{Машинки}
\indent\parЭта задача, пожалуй, самая зрелищная из всех примеров. Постоянной величиной здесь является случайно сгенерированная трасса, представленная ломаной. В качестве индивида здесь выступает машина. Машина представляет собой набор векторов из одной очки, на которые «натянут» многоугольник, называемый «корпусом».  В некоторых вершинах этого многоугольника находятся колёса, которые соединены с вершиной с помощью пружины. ДНК машины хранит длину каждого из векторов, радиус каждого из колёс, его наличие либо отсутствие в конкретной вершине, а также жёсткость соединительной пружины. При этом для каждого из параметров пользователь задаёт минимальное и максимальное значение. Примеры машин из разных поколений вы можете увидеть на рисунках, расположенных ниже:

\begin{center}
\includegraphics[bb = 0 0 1366 768,scale=0.4]{picture1.png}
\end{center}

\begin{center}
\includegraphics[bb = 0 0 1366 768,scale=0.4]{picture2.png}
\end{center}
\newpage
\section{Заключение}
\parВ результате нашей работы, нами были:
\begin{itemize}
\item Изучены генетические алгоритмы.
\item Разработана библиотека.
\item Библиотека была успешно интегрирована в ряде приложений.
\item Была изучена эффективность использования ГА.
\end{itemize}
В дальнейшем мы планируем:
\begin{itemize}
\item Добавить новые компоненты в библиотеку и улучшить уже присутствующие.
\item Оптимизировать библиотеку.
\end{itemize}

\section{Используемая литература}
\begin{enumerate}
\item Панченко, Т.В. Генетические алгоритмы \slash{} Т.В.Панченко. Издательский дом ''Астраханский университет'' 2007.
\item Booker, L.B. Classifier Systems and Genetic Algorithms. \slash{} L.B. Booker, D.E. Goldbergand, J.H. Holland.
\item Kenneth De Jong. Genetic Algorithm Based Learning.
\item Kenneth De Jong. Genetic Algorithms Are NOT Function Optimizers.
\item Бедный, Ю.Д. Применение генетических алгоритмов для генерации
автоматов при построении модели максимального правдоподобия и в задачах управления. \slash{} Ю.Д. Бедный, Магистерская диссертация, Санкт-Петербург, 2008.
\end{enumerate}

\end{document}
